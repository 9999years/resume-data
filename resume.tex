\documentclass{article}
\usepackage{resume}
\begin{document}
\maketitle

% TODO be consistent with periods in langs, and capitalization in position
% names
% TODO be consistent with "and"s

%\begin{itemize}
	%\item Sophomore computer science \bs\ and mathematics \ba\ at
	%Brandeis University.
	%%\item Preferred title: Software Engineer Intern.
	%\item Available early May through late August.
%\end{itemize}

%I am a sophomore computer science student enrolled at Brandeis University
%passionate about software engineering and art, including graphic design and
%motion graphics. I'm looking for an internship in software development where
%I can contribute meaningfully and learn real-world development practices.

%I'm looking for part-time library science work where I can
%connect the greater community to the services a library provides.

%I'm looking for part-time tutoring work where I can apply my unique skills
%and extensive knowledge of math and science to help kids learn and thrive in
%schools that don't or aren't able to provide a nurturing or engaging
%atmosphere.

%I'm looking for part-time office or administrative work where I can use my
%outgoing and friendly personality to make customers feel welcome and apply
%my extensive knowledge of computers and electronics to efficiently help
%customers find what they need.

\section{Professional experience}

\begin{job}[company=Google,
	city={Mountain View, \textsc{ca}},
	dates=June 2019--August 2019,
	title=Software Engineer Intern,
	]
\item Migrated settings panels between frameworks and added extra settings
to control spam filtering.
\item Fixed old, unmaintained build and test targets.
\end{job}

%\begin{job}[company={Computer Science Department, Brandeis University},
	%dates=August 2019--December 2019,
	%title=Operating Systems Teaching Assistant,
	%list=false, ]
%\end{job}

\begin{job}[company={Brandeis University},
	city={Waltham, \textsc{ma}},
	dates=February 2019--, title=Systems Administrator,
	supervisor=Christopher Allison,
	supervisorcontact=\email{chris@cs.brandeis.edu},
	]
\item Replaced a manual macOS deployment process and undocumented Bash
scripts with Ansible.
\item Migrated legacy documentation website into a modern framework while
rewriting and verifying old information. \langs{Hugo, Markdown, \scss,
\html}
\end{job}

\begin{job}[samecompany,
	city={Waltham, \textsc{ma}},
	dates=September 2018--January 2019, title=Lead Software Engineer,
	supervisor=Priyanka Renugopalakrishnan,
	supervisorcontact=\email{priyankarina@brandeis.edu}]

\item Increased uptime and reliability by documenting and automating
an error-prone process. \langs{Bash}

\item Improved security by removing insecure language features from
the codebase, migrated 100,000~\sloc\ from Python~2 to Python~3.

\item Expanded documentation from a single file into its own repository,
reorganizing it into an
\href{https://github.com/rust-lang-nursery/mdBook}{mdBook}. Wrote a Django
plugin to integrate documentation with main site. \langs{Markdown, mdBook,
Python, Django}

%\item Recovered from legacy source code losses by manually repairing an
%internal database table and regenerating schema changes. \langs{Python,
%PostgreSQL}

%\item Revised and rewrote security policy.

\end{job}

\begin{job}[company=Iridium, dates=May--August 2018, title=Engineering Intern,
	city={Leesburg, \textsc{va}},
	supervisor=Ken Rock, supervisorcontact=\email{Ken.Rock@iridium.com}]

\item Authored tools to detect anomalies within satellite configurations by
searching for mismatches.
%Generated reports in \html\ and plain-text and
%sent automated emails to relevant engineers. Migrated several components
%into generic organization-wide libraries.
\langs{Python, Perl, \html, Bash}

\item Created a tool to analyze processor memory dumps in order to simplify
memory error correction.
\langs{Python}

\item Created an interactive MongoDB query builder for data science staff.
Provided rich input based on each field's data type.
\langs{C++, Qt}

\end{job}

\section{Education}

\begin{education}
\edu[dates=August 2017--May 2021 (expected),
	degree={Computer science \bs\ and mathematics \ba}, gpa=3.644,
	location={Waltham, \ma}]{Brandeis University}

Courses include operating systems, advanced programming techniques, the
theory of computation, the structure and interpretation of computer
programs, data structures, calculus, linear algebra, proofs, and physics
(mechanics / electricity and magnetism).

Dean's list Fall 2017, Fall 2018, Spring 2019.

\edu[dates=June 2017, degree=Honors Diploma in computer science,
	location={Fairfax, \va}]{The New School of Northern Virginia}
\end{education}

\section{Software}

Extensive experience: Java, Python, C, JavaScript, PowerShell, LaTeX,
\php, \html\ 5, \css\ 3, Perl, Bash, and Go.

Moderate experience: Ruby, Rust, Wolfram Mathematica, C++, Haskell,
Racket / Scheme, C\Sharp, F\Sharp, the \dotnet\ \api, MongoDB, PostgreSQL,
Redis, Qt, Django, TypeScript.

Comfortable working on Linux, macOS, and Windows.

\subsection{Open Source}

I have published packages on PyPI (Python) as
\href{https://pypi.org/user/9999years/}{\Tt{9999years}} and \ctan\ (LaTeX)
as \href{https://ctan.org/author/turner}{\Tt{turner}}. I have contributed
code and documentation to a variety of open source projects; pull requests
include \pr{pypa/packaging.python.org}{457}, \pr{pypa/sampleproject}{66},
\pr{gogs/gogs}{5126}, \pr{russross/blackfriday}{453},
\pr{abiosoft/caddy-git}{88}, \pr{new-xkit/XKit}{1653},
\pr{crdoconnor/strictyaml}{55}, \pr{lervag/vimtex}{1358},
\pr{azavea/ansible-pip}{14}, and \pr{dahlbyk/posh-git}{277}.

\subsection{Independent projects}

I have developed a number of independent software projects, including:

\begin{softwarelist}

\begin{software}{name=I C the Light, github=9999years/i-c-the-light,
		langs=Written in C}
	a distance-estimating ray marcher designed to render images of
	quaternion Julia sets. \\
	Paper: \link{https://becca.ooo/i-c-the-light.pdf}{becca.ooo/i-c-the-light.pdf}
\end{software}

\begin{software}{name=juliaplotter, github=9999years/juliaplotter, tt,
		langs=Python}
	a script to render cells or grids of arbitrary Julia sets, to
	overcome difficulties other renderers have in visualizing the many
	possible values of the constant \It{c} in the Julia set for a given
	rational function.
\end{software}

\newcommand{\hashwrap}{\link{https://chrome.google.com/webstore/detail/wrap/nbcgkdilbhnnoemimofnknocbkpldobi}{\#wrap}}
\begin{software}{name=\#wrap,
		link=https://chrome.google.com/webstore/detail/wrap/nbcgkdilbhnnoemimofnknocbkpldobi,
		github=9999years/hashwrap,
		langs=\css}
	an extension for Google Chrome to wrap Tumblr's hashtag interface to
	one line, addressing a widely-held annoyance with Tumblr's user
	interface. Three years after its publication, \namedlink\ had
	retained 30,315 active users.
\end{software}


\begin{software}{simplelink=becca.ooo/mandelbrot,
		langs={\Sc{html 5} canvas, JavaScript}}
	an interactive web app to explore the Mandelbrot set and visualize
	incremental values of the underlying sequence at arbitrary points.
\end{software}

\begin{software}{name=Twitter bots, github=9999years/twitter-bots,
		langs=\php, comma=false}
	such as \twitter{gardensbot}, which posts grids of plant and animal
	emoji in virtual ``gardens'' every other hour, and
	\twitter{goodwordsbot}, which generates fake but plausible English
	words from Latin and Greek roots.
\end{software}

\begin{software}{name=The Daily Report, github=9999years/daily-report,
		langs=Python}
	a highly-customizable framework for generating daily agendas for use
	with receipt printers. Capable of reading and processing Google
	calendars, outputting weather forecasts, countdowns, to-dos, daily
	news, and other useful information. Currently running on a Raspberry
	Pi interfacing with a Star thermal receipt printer.
\end{software}

%\item \link{https://github.com/9999years/PSVarDump/}{PSVarDump}, a
%PowerShell module for debugging PowerShell scripts by inspecting variable
%contents. \\
%PowerShell script. \github{9999years/PSVarDump}

\begin{software}{simplelink=becca.ooo/fox,
		langs={\Sc{html 5} canvas, JavaScript}}
	a program to simulate and visualize the interactions of plant and
	animal populations in a virtual ecosystem.
\end{software}

%\item \simplelink{glitcher.becca.ooo}, an
%in-browser databending app which procedurally glitches images by modifying
%their binary data. \\
%\php, JavaScript.

%\begin{software}{simplelink=becca.ooo/integral,
		%langs={\Sc{html 5} canvas, JavaScript}}
	%an interactive web app to visualize integral approximations.
%\end{software}

\end{softwarelist}

%\section{Graphic design / motion graphics}

%I have experience with Adobe \Sc{cc} suite software, and Cinema \Sc{4d}
%across a variety of fields --- I have created vector illustrations,
%manipulated photos and videos, and created motion graphics from scratch.

%A \Sc{Wired} article
%(\link{https://www.wired.com/2014/10/11-beautiful-psychedelic-gifs-created-math-whiz/}
%{11 Beautiful, Psychedelic GIFs Created by a Math Whiz}) linked to a
%directory of \Sc{gif} artists I curate, \simplelink{gifartists.tumblr.com}.
%Motion graphics reel available at \simplelink{becca.ooo/reel}

%Portfolio available on request.

%I have created and posted \Sc{3d} animated \gif s daily on my blog at
%\link{https://9999yea.rs/}{9999yea.rs} and gathered more than 15,000
%followers.

%I shoot, develop, scan, and edit my own 35mm film.
%\link{https://www.flickr.com/photos/techb/}{flickr.com/techb}

%I have spent years practicing and refining my skills at digital art
%(including graphic design, \Sc{3d} animation, motion graphics, and
%post-processing) and calligraphy to be truly skilled artistically and
%creatively.

%I can grapple with high-level mathematical concepts and work until I
%have an intuitive understanding of complicated proofs.

%\section{Leadership}

%I have served on the Fairness Committee to smoothly and healthily
%resolve various interpersonal issues, with the goal of promoting
%understanding and happiness in the wider community.

%Additionally, I manage a brief daily meeting of 130 people.

\section{Publications}

Co-author of
\href{https://www.unicode.org/L2/L2019/19025-terminals-prop.pdf}{``Proposal
to add characters from legacy computers and teletext to the UCS''}
(L2/19-025, 2019), accepted at \ac{utc} 158 ``for a future version of
Unicode.''

\section{Volunteering}

Speed mentoring for 4-H youth, resume-building with the African American
Community Service Agency.

\end{document}
